% Copyright (c) 2022 by Lars Spreng
% This work is licensed under the Creative Commons Attribution 4.0 International License. 
% To view a copy of this license, visit http://creativecommons.org/licenses/by/4.0/ or send a letter to Creative Commons, PO Box 1866, Mountain View, CA 94042, USA.

%~~~~~~~~~~~~~~~~~~~~~~~~~~~~~~~~~~~~~~~~~~~~~~~~~~~~~~~~~~~~~~~~~~~~~~~~~~~~~~
% You can add your packages and commands to the loadslides.tex file. 
% The files in the folder "styles" can be modified to change the layout and design of your slides.
% I have included examples on how to use the template below. 
% Some of it these examples are taken from the Metropolis template.
%~~~~~~~~~~~~~~~~~~~~~~~~~~~~~~~~~~~~~~~~~~~~~~~~~~~~~~~~~~~~~~~~~~~~~~~~~~~~~~

\documentclass[
11pt,notheorems,hyperref={pdfauthor=whatever}
]{beamer}


% Copyright (c) 2022 by Lars Spreng
% This work is licensed under the Creative Commons Attribution 4.0 International License. 
% To view a copy of this license, visit http://creativecommons.org/licenses/by/4.0/ or send a letter to Creative Commons, PO Box 1866, Mountain View, CA 94042, USA.

%~~~~~~~~~~~~~~~~~~~~~~~~~~~~~~~~~~~~~~~~~~~~~~~~~~~~~~~~~~~~~~~~~~~~~~~~~~~~~~
% Add your packages and commands to this file
%~~~~~~~~~~~~~~~~~~~~~~~~~~~~~~~~~~~~~~~~~~~~~~~~~~~~~~~~~~~~~~~~~~~~~~~~~~~~~~

%~~~~~~~~~~~~~~~~~~~~~~~~~~~~~~~~~~~~~~~~~~~~~~~~~~~~~~~~~~~~~~~~~~~~~~~~~~~~~~
\RequirePackage{palatino}
\RequirePackage[utf8]{inputenc}
\RequirePackage[T1]{fontenc}

\usefonttheme{serif}

\usepackage{styles/elegantmacros}
\usefolder{styles}
\usetheme[style=blue]{elegant}

\newcommand{\makepart}[1]{ % For convenience
\part{#1} \frame{\partpage}
} 

%~~~~~~~~~~~~~~~~~~~~~~~~~~~~~~~~~~~~~~~~~~~~~~~~~~~~~~~~~~~~~~~~~~~~~~~~~~~~~~

%~~~~~~~~~~~~~~~~~~~~~~~~~~~~~~~~~~~~~~~~~~~~~~~~~~~~~~~~~~~~~~~~~~~~~~~~~~~~~~
% Figures
\RequirePackage{booktabs}
\RequirePackage{colortbl}
\RequirePackage{ragged2e}
\RequirePackage{schemabloc}
%\RequirePackage{natbib}
\RequirePackage{caption}
\RequirePackage{subcaption}
\RequirePackage{tabularx}
\RequirePackage{array}
\RequirePackage{multirow}
\usepackage[
  style=authoryear, 
]{biblatex}
\RequirePackage{xcolor}
\addbibresource{references.bib}
\newcolumntype{Y}{>{\centering\arraybackslash}X}

%~~~~~~~~~~~~~~~~~~~~~~~~~~~~~~~~~~~~~~~~~~~~~~~~~~~~~~~~~~~~~~~~~~~~~~~~~~~~~~

%~~~~~~~~~~~~~~~~~~~~~~~~~~~~~~~~~~~~~~~~~~~~~~~~~~~~~~~~~~~~~~~~~~~~~~~~~~~~~~
% Figures
\RequirePackage{wrapfig}
\RequirePackage{pgfplots}
\RequirePackage{graphicx}
\RequirePackage{adjustbox}
\RequirePackage{environ}
\pgfplotsset{compat=1.18  }

\makeatletter
\newsavebox{\measure@tikzpicture}
\NewEnviron{scaletikzpicturetowidth}[1]{%
  \def\tikz@width{#1}%
  \def\tikzscale{1}\begin{lrbox}{\measure@tikzpicture}%
  \BODY
  \end{lrbox}%
  \pgfmathparse{#1/\wd\measure@tikzpicture}%
  \edef\tikzscale{\pgfmathresult}%
  \BODY
}
\makeatother
%~~~~~~~~~~~~~~~~~~~~~~~~~~~~~~~~~~~~~~~~~~~~~~~~~~~~~~~~~~~~~~~~~~~~~~~~~~~~~~

%~~~~~~~~~~~~~~~~~~~~~~~~~~~~~~~~~~~~~~~~~~~~~~~~~~~~~~~~~~~~~~~~~~~~~~~~~~~~~~
% Maths 
\RequirePackage{textcomp}
\RequirePackage{amsmath} 
\RequirePackage{amsthm}
\RequirePackage{mathtools}
\RequirePackage{braket}
%\RequirePackage{bbm}
%\RequirePackage{algorithm}
%\RequirePackage[osf,sc]{mathpazo}
%\RequirePackage{pifont}
%\newcommand{\xmark}{\ding{55}}%
%\numberwithin{equation}{section}
\DeclareMathOperator*{\argmax}{arg\,max}
\DeclareMathOperator*{\argmin}{arg\,min}

\setbeamertemplate{theorems}[numbered] % to number

\theoremstyle{definition}
\newtheorem{fact}{Fact}[section]
\newtheorem{examp}{Example}[section]

\theoremstyle{plain}
\newtheorem{definition}{Definition}[section]
\newtheorem{proposition}{Proposition}
\newtheorem{theorem}{Theorem}
\newtheorem{assumption}{Assumption}

\providecommand{\H}{\mathscr{H}}      
\providecommand{\E}{\mathbb{E}}
\makeatletter
\def\munderbar#1{\underline{\sbox\tw@{$#1$}\dp\tw@\z@\box\tw@}}
\makeatother

%~~~~~~~~~~~~~~~~~~~~~~~~~~~~~~~~~~~~~~~~~~~~~~~~~~~~~~~~~~~~~~~~~~~~~~~~~~~~~~
%Commands
\newcommand{\importanttext}[1]{\Large #1 \normalsize}
\newcommand{\crt}[1]{a_{#1}^{\dagger}}
\newcommand{\ani}[1]{a_{#1}}
\newcommand{\vac}{\ket{0}}
\newcommand{\gs}{\ket{\Phi_0}}
\newcommand{\exed}[2]{\ket{\Phi_{#1}^{#2}}}

\newcommand{\inner}[3]{\bra{#1}#2\ket{#3}}
\newcommand{\innerAS}[3]{\inner{#1}{#2}{#3}_{\text{AS}}}
\newcommand{\mathset}[1]{\{ #1 \}}
\newcommand{\ord}[2]{#1^{(#2)}}
\newcommand{\energyref}{E_0^{\text{ref}}}
\newcommand{\psum}{\sideset{}{'}\sum}

\newcommand{\spinop}[1]{\hat{S}_{#1}}
\newcommand{\pplus}[1]{\hat{P}_{#1}^{+}}
\newcommand{\pminus}[1]{\hat{P}_{#1}^{-}}
\newcommand{\numberop}[1]{\hat{N}_{#1}}
\newcommand{\hatreefock}[1]{#1^{\text{HF}}}
\newcommand{\comment}[1]{\textcolor{red}{#1}}

% Colors
\definecolor{Green}{RGB}{20, 128, 10}
\definecolor{Red}{RGB}{153, 22, 8} % Loads packages and some defined commands

\title[
% Text entered here will appear in the bottom middle
]{FYS4480 Oral exam, midterm one and two}

\subtitle{Helium and Beryllium using CIS and Hartree-Fock, pairing model}

\author[
% Text entered here will appear in the bottom left corner
]{
    Håkon Kvernmoen
}

\institute{
    University of Oslo}
\date{\today}

\begin{document}

% Generate title page
{
\setbeamertemplate{footline}{} 
\begin{frame}
  \titlepage
\end{frame}
}
\addtocounter{framenumber}{-1}

% You can declare different parts as a parentof sections
% \begin{frame}{Part I: Demo Presentation Part}
%     \tableofcontents[part=1]
% \end{frame}
% \begin{frame}{Part II: Demo Presentation Part 2}
%     \tableofcontents[part=2]
% \end{frame}

% \makepart{Demo Part}

% \section{Introduction}
% \begin{frame}
% \begin{itemize}
%     \item This template provides an elegant and minimalistic layout for beamer slides. Hence the name \alert{\textbf{Elegant Slides}}.
%     \item I created Elegant Slides because I wasn't satisfied with any of the existing Beamer templates, which look slightly different than Elegant Slides.
%     \item My goal was to create a layout that is \alert{\textbf{simplistic but beautiful}} and focuses on the content, rather than crowding each slide with lots of different coloured boxes.
%     \item I designed Elegant Slides for \alert{\textbf{lecture notes and technical presentations}} but it can be used for any kind of talk. 
% \end{itemize}
     
% \end{frame}

\begin{frame}
    \frametitle{Setup}
    Represent states using creation $\crt{p}$ and annihilation $\ani{q}$ operators (occupation representation/second quantization), obeying
    \begin{align*}
        \mathset{\crt{p},\ani{q}} = \delta_{pq},\hspace{20px}\mathset{\crt{p},\crt{q}} = \mathset{\ani{p},\ani{q}} = 0
    \end{align*}
    $p$ and $q$ are sets of relevant quantum numbers. We need to pick a single particle (SP) computational basis, having $n$ possible single particle states.
    \begin{align*}
        \text{3D HO:}\hspace{20px}p = \mathset{n_r,l,m_l,s,m_s},\hspace{20px} \crt{p} \vac = \ket{p} \longrightarrow \psi_p(\bold{x}) = \psi_{n_r l m}(r, \theta, \phi) = \ldots 
    \end{align*} 
    Need a Hamiltonian to solve $\hat{H} = \hat{H}_0 + \hat{V}$, with $\hat{H}_0$ representing single particle energy contributions, and $\hat{V}$ interactions.
    \\[10pt]
    Often start with an $N$-particle ground state ansatz $\ket{\Phi_0} = \crt{1},\ldots \crt{N} \vac$.
    \\[10pt]
    And consider excitations of this
    \begin{align*}
        \text{1p1h}\hspace{20px}&\exed{i}{a} = \crt{a}\ani{i}\gs \\
        \text{2p2h}\hspace{20px}&\exed{ij}{ab} = \crt{a}\crt{b}\ani{j}\ani{i}\gs  \\
        \text{NpNh}\hspace{20px}&\exed{ij\ldots}{ab\ldots} = \crt{a}\crt{b}\ldots\ani{j}\ani{i}\gs 
    \end{align*}
\end{frame}


\subsection{FCI}
\begin{frame}
    \frametitle{Full configuration interaction (FCI)}

    Start with $N$ particle ground state ansatz $\ket{\Phi_0}$.
    \\[10pt]
    Not an eigenstate of $\hat{V}$ and therefor not the true ground state of the system.
    \\[10pt]
    By considering every possible $N$ particle state in our system (using $\mathset{\crt{1},\ldots,\crt{N},\ldots,\crt{n}}$), we can construct our ground state $\ket{\Psi_0}$ as a linear combination of excited states.
    \begin{align*}
        \ket{\Psi_0} = C_0 + \sum_{ai} C_i^a \exed{i}{a} + \sum_{ai} C_{ij}^{ab} \exed{ij}{ab} + \ldots
    \end{align*} 
    \\[10pt] 
    Normally solved by considering the Hamiltonian in matrix representation, with elements $H_{XY} = \inner{\Phi_X}{\hat{H}}{\Phi_Y}$, with $X, Y \in \mathset{\text{0p0h},\text{1p1h}\ldots\text{NpNh}}$, giving the eigenvalue problem 
    \begin{align*}
        H \bold{c} = E\bold{c}
    \end{align*}
    When solved, the smallest eigenvalue $\ord{E}{0}$ will yield the ground state energy, and $\ket{\Psi_0}$ can be found by considering the eigenvectors $\ord{\bold{c}}{0} = (C_0, C_{i}^{a} \ldots C_{ij}^{ab} \ldots \ldots C_{ij\ldots}^{ab\ldots})$. Excited states can also be found by considering the other eigenvalues and vectors $\ord{E}{i}, \ord{\bold{c}}{i}$.
\end{frame}

\subsection{FCI}
\begin{frame}
    Example, pairing interaction: $\hat{V}$ only works between spin-paired states at the same energy level. We let $p = 1, 2, 3, 4$ denote energy levels, with SP states also having a spin $\sigma = \pm$, a total of $n = 8$ SP states. 
    \begin{align*}
        \hat{H} = \hat{H}_0 + \hat{V},\hspace{20px} \hat{H}_0 = \sum_{p\sigma} (p-1)\crt{p\sigma}\ani{p\sigma}, \hspace{20px}\hat{V} = -\frac{1}{2}g\sum_{pq}\crt{p+}\crt{p-}\ani{q-}\ani{q+}
    \end{align*}
    Diagonal SP Hamiltonian $\hat{H}_0$ and two body interaction $\hat{V}$. Consider $N = 4$ particles with total spin $S = 0$, writing $\ket{PQ} = \ket{p+p-q+q-}$ we have a total of six different many body states.
    \begin{align*}
        \text{0p0h}&\hspace{20px} \ket{12}  \\
        \text{2p2h}&\hspace{20px} \ket{13}, \ket{14}, \ket{23}, \ket{24} \\  
        \text{4p4h}&\hspace{20px} \ket{34}
    \end{align*}
    By setting up the matrix $\inner{KL}{\hat{H}}{RS}$, we get a small eigenvalue problem of a $6 \times 6$ matrix.
\end{frame}

\subsection{FCI}
\begin{frame}
    \begin{figure}
        \centering
        \includegraphics[width=0.75\linewidth]{figs/FCI.pdf}
    \end{figure}
    But there is a problem...
\end{frame}

\subsection{FCI}
\begin{frame}
    In all but very simple problems, this approach is unfeasible. In general, we have to consider 
    \begin{align*}
        \binom{n}{N} = \frac{n!}{N!(n-N)!}
    \end{align*}
    many body states. Taking our pairing model example, lifting the $S = 0$ restriction yields 70 different states. This is still possible, but increasing both $n$ and $N$ results in disaster
    
    \begin{table}
        \begin{tabular}{c|l|l|l|l}
        $N \downarrow /n \rightarrow$ & 8    & 32       & 64        & 128       \\
        \hline
        4     & $70$ & $10^{4}$ & $10^{5}$  & $10^{7}$  \\
        8     &      & $10^{7}$ & $10^{9}$  & $10^{12}$ \\
        16    &      & $10^{8}$ & $10^{14}$ & $10^{19}$ \\
        32    &      &          & $10^{18}$ & $10^{30}$
        \end{tabular}
        \caption{NB: Order of magnitude values}
    \end{table}
\end{frame}

\subsection{FCI}
\begin{frame}
    \textcolor{Green}{\importanttext{Pros:}}
    \begin{itemize}
        \item Provides exact solutions within a truncated basis set
        \item Understandable and relatively easy to set up 
        \item Excited states thrown into the bargain
    \end{itemize}
    \vspace{20px}
    \textcolor{Red}{\importanttext{Cons:}}
    \begin{itemize}
        \item Computational complexity, bad scaling
        \item Only possible for tiny systems, with few states and particles.
        \item Practically only a benchmarking tool
    \end{itemize}
\end{frame}

\begin{frame}{}{Custom Subsection}
    This frame has a custom subtitle. The frame title is automatically inserted and corresponds to the section title.
\end{frame}

\begin{frame}{Custom Title}{Custom Subsection with Footnote}
    This frame has a custom title and a custom subtitle.\footnote{This is a footnote. See also \textcite{example_2022}. }
\end{frame}

\subsection{Typographics}
\begin{frame}
    These examples follow the Metropolis Theme
    \begin{itemize}
        \item Regular
        \item \alert{Alert}
        \item \textit{Italic}
        \item \textbf{Bold}
    \end{itemize}
\end{frame}

\subsection{Lists}

\begin{frame}
    \begin{columns}[T,onlytextwidth]
    \column{0.33\textwidth}
      \textbf{Items}
      \begin{itemize}
        \item Cats 
        \begin{itemize}
            \item British Shorthair
        \end{itemize}
        \item Dogs \item Birds
      \end{itemize}

    \column{0.33\textwidth}
      \textbf{Enumerations}
      \begin{enumerate}
        \item First 
        \begin{enumerate}
            \item First subpoint
        \end{enumerate}
        \item Second \item Last
      \end{enumerate}

    \column{0.33\textwidth}
      \textbf{Descriptions}
      \begin{description}
        \item[Apples] Yes \item[Oranges] No \item[Grappes] No
      \end{description}
\end{columns}
\end{frame}

\subsection{Table}
\begin{frame}
    \begin{table}
        \caption{Largest cities in the world (source: Wikipedia)}
        \begin{tabular}{@{} lr @{}}
          \toprule
          City & Population\\
          \midrule
          Mexico City & 20,116,842\\
          Shanghai & 19,210,000\\
          Peking & 15,796,450\\
          Istanbul & 14,160,467\\
          \bottomrule
        \end{tabular}
        \hspace*{1cm}
            \setlength\extrarowheight{3pt}
        \begin{tabular}{|lr|}
          \hline
          \rowcolor{primary}\color{white}City & \color{white}Population\\
          \hline
          Mexico City & 20,116,842\\
          Shanghai & 19,210,000\\
          Peking & 15,796,450\\
          Istanbul & 14,160,467\\
          \hline
        \end{tabular}
    \end{table}
\end{frame}

\subsection{Figures}
\begin{frame}
    \begin{figure}[htbp]
        \centering
        \caption{Plot of $y=x^2$}
        \begin{tikzpicture}
            \begin{axis}[
            legend columns=3,
            legend style={at={(0.5,-0.3)},anchor=north},
            width = \textwidth,
            height = 2.5in,
            xmin = -3, 
            xmax = 3,
            ymin = 0,
            ymax = 10,
            ]
                \addplot[primary] {x^2};
                        \addlegendentry{$x^2$}
            \end{axis}
        \end{tikzpicture}
    \end{figure}

\end{frame}

\subsection{Blocks}
\begin{frame}

   \centering
	\begin{minipage}[b]{0.5\textwidth}

	  \begin{block}{Default}
        Block content.
      \end{block}

      \begin{alertblock}{Alert}
        Block content.
      \end{alertblock}

      \begin{exampleblock}{Example}
        Block content.
      \end{exampleblock}      
      
	\end{minipage}	
\end{frame}

\section{Maths}
\subsection{Equations}
\begin{frame}
    \begin{itemize}
        \item A numbered equation:
        \begin{equation}
            y_t = \beta x_t + \varepsilon_t
        \end{equation}
         \item Another equation:
        \begin{equation*}
            \mathbf{Y} = \boldsymbol{\beta} \mathbf{X} + \boldsymbol{\varepsilon}_t
        \end{equation*}
    \end{itemize}
\end{frame}

\subsection{Theorem}
\begin{frame}
    \begin{itemize}
        \item Theorems are numbered consecutively.
    \end{itemize}
    \begin{theorem}[Example Theorem]
         Given a discrete random variable X, which takes values in the alphabet $\mathcal{X}$ and is distributed according to  $p:{\mathcal {X}}\to [0,1]$:
            \begin{equation}
                \mathrm {H} (X):=-\sum _{x\in {\mathcal {X}}}p(x)\log p(x)=\mathbb {E} [-\log p(X)]
            \end{equation}
    \end{theorem}
\end{frame}

\begin{frame}{}{Definitions}
    \begin{itemize}
        \item Definition numbers are prefixed by the section number in the respective part.
    \end{itemize}
     \begin{definition}[Example Definition]
         Given a discrete random variable X, which takes values in the alphabet $\mathcal{X}$ and is distributed according to  $p:{\mathcal {X}}\to [0,1]$:
            \begin{equation}
                \mathrm {H} (X):=-\sum _{x\in {\mathcal {X}}}p(x)\log p(x)=\mathbb {E} [-\log p(X)]
            \end{equation}
    \end{definition}
\end{frame}

\begin{frame}{}{Examples}
    \begin{itemize}
        \item Examples are numbered as definitions.
    \end{itemize}
    \begin{examp}[Example Theorem]
         Given a discrete random variable X, which takes values in the alphabet $\mathcal{X}$ and is distributed according to  $p:{\mathcal {X}}\to [0,1]$:
            \begin{equation}
                \mathrm {H} (X):=-\sum _{x\in {\mathcal {X}}}p(x)\log p(x)=\mathbb {E} [-\log p(X)]
            \end{equation}
    \end{examp}
\end{frame}

\makepart{Demo Presentation Part 2}

\section{Section}
\subsection{Subsection}
\subsubsection{Subsubsection}
\subsubsection{Subsubsection}
\subsection{Subsection}
\subsubsection{Subsubsection}
\subsubsection{Subsubsection}
\section{Section}
\subsection{Subsection}
\subsubsection{Subsubsection}
\subsubsection{Subsubsection}
\subsection{Subsection}
\subsubsection{Subsubsection}
\subsubsection{Subsubsection}

\begin{frame}[allowframebreaks]{References}
    \printbibliography
\end{frame}
\end{document}