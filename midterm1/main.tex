\documentclass{article}
\usepackage[utf8]{inputenc}
\usepackage[margin=0.6in]{geometry}
% Math
\usepackage{amsmath}
\usepackage{physics}
\usepackage{braket}
\usepackage{xfrac}

% Citetations
\usepackage[ backend=bibtex, sorting=none, autocite=plain]{biblatex}
\addbibresource{refs/references}
\usepackage{xcolor}
\usepackage{hyperref}
\hypersetup{
    colorlinks,
    linkcolor={red!50!black},
    citecolor={blue!50!black},
    urlcolor={blue!80!black}}


% Formatting
\usepackage{float}
\usepackage{graphicx}
\graphicspath{ {./figs/} } 

% Misc
\usepackage{appendix}

% Commands
\newcommand{\vac}{\ket{0}}
\newcommand{\gs}{\ket{\Phi_0}}
\newcommand{\exed}[2]{\ket{\Phi_{#1}^{#2}}}

\newcommand{\ups}[1]{#1\uparrow}
\newcommand{\downs}[1]{#1\downarrow}

\newcommand{\inner}[3]{\braket{#1|#2|#3}}
\newcommand{\innerAS}[3]{\inner{#1}{#2}{#3}_{\text{AS}}}

\newcommand{\hnull}{\hat{h}_0}

\newcommand{\crt}[1]{a_{#1}^{\dagger}}
\newcommand{\ani}[1]{a_{#1}}


\title{Interesting title}
\author{Håkon Kvernmoen}
\date{MONTH YEAR}

\begin{document}
\maketitle

\subsection*{Part a), setting up the basis.}
    Since we use hydrogen-like single particle states, we consider the quantum numbers $nlm_l sm_s$. We constrain ourselves to only look at $n = 1,2,3$, with $l = 0$ implying $m_l = 0$ In addition, since electrons are fermions, all single particle states (SPS) must have $s = \sfrac{1}{2}$, giving two possible spin projections $m_s = \sfrac{1}{2}, -\sfrac{1}{2}$, which we will write as $\uparrow, \downarrow$ respectively. Due to these restrictions, the only relevant quantum numbers are $nm_s$.       
    
    We begin by setting up an ansatz (educated guess) for the helium ground state $\gs$. The one body energy $\inner{i}{\hnull}{i}$ is increasing with $n$, thus it makes sense that the lowest energy multi particle state (MPS) should prioritize the $n = 1$ SPS. No two fermions can be in the same state, thus setting both electrons in $n = 1$ requires antiparallel spin. Thus, using the creation and annihilation operators ($\crt{nm_s}, \ani{nm_s}$) from the second quantization formalism, we conclude:   
    \begin{align*}
        \gs = \crt{\ups{1}}\crt{\downs{1}}\vac
    \end{align*}
    With $\vac$ representing the vacuum state. Working with $\vac$ becomes cumbersome when we wish to look at excitations of $\gs$, especially if $\gs$ contains many electrons. Therefor we "redefine the vacuum", that is set $\gs$ as the Fermi level. This means that we will add and remove particles from $\gs$ to represent excited states, instead of constructing these from $\vac$. This introduces the language of \textit{particles} and \textit{holes}. A particle state is a filled state above the Fermi level, that is adding a $n = 2$ or $3$ state to $\gs$. On the contrary a hole state is a non-filled state below the Fermi level, that is removing one of the $n = 1$ electrons from $\gs$. We will refer to hole states using letters $i,j,k, ...$ and particle states using letters $a,b,c,...$.          

\subsection*{Part b), Second quantized Hamiltonian.}
\subsection*{Part c), Limiting ourselves to one-particle-one-hole excitations.}
\subsection*{Part d), Moving to the Beryllium atom.}
\subsection*{Part e), Hartree-Fock.}
\subsection*{Part f), The Hartree-Fock matrices.}
\subsection*{Part g), Writing a Hartree-Fock code.}


\begin{appendix}
    \section{Appendix entry}
    some appendix things
\end{appendix}
\cite{DFTgap}
\printbibliography

\end{document}

